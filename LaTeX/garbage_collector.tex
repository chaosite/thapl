\documentclass[t,notheorems,compress]{beamer}

\setbeamersize{text margin left=1em,text margin right=1em}


\usepackage{tikz}
\usepackage{bashful}
\usepackage{adjustbox}

\adjustboxset{max width=\columnwidth}
\adjustboxset{max height=0.8\textheight}
\adjustboxset{center}

\usetikzlibrary{arrows}
\usetikzlibrary{arrows.meta}
\usetikzlibrary{calc}
\usetikzlibrary{chains}
\usetikzlibrary{circuits.ee.IEC}
\usetikzlibrary{decorations}
\usetikzlibrary{decorations.pathreplacing}
\usetikzlibrary{decorations.text}
\usetikzlibrary{graphs}
\usetikzlibrary{mindmap}
\usetikzlibrary{patterns}
\usetikzlibrary{positioning}
\usetikzlibrary{shadows}
\usetikzlibrary{shapes}
\usetikzlibrary{shapes.callouts}
\usetikzlibrary{shapes.multipart}
\usetikzlibrary{tikzmark}

\newcounter{offset}

\tikzstyle{GREEN}=[fill=green!50!black]

\tikzset{%
  invisible/.style={opacity=0},
  visible on/.style={alt={#1}},
  alt/.code args={<#1>}{%
    \alt<#1>{}{\pgfkeysalso{invisible}} % \pgfkeysalso doesn't change the path
  },
}

\tikzset{val/.style={draw,fill=blue!30,font=\scriptsize\ttfamily}}
\tikzset{var/.style={draw,rectangle,fill=red!30,minimum width=0.5cm,minimum height=0.5cm,font=\scriptsize}}

\tikzset{CONS/.style= {val,SPLITh=2}}
\tikzset{CONSv/.style={val,SPLITv=#1}}
\tikzset{CONSh/.style={val,SPLITh=#1}}
\tikzset{SPLITh/.style={SPLIT=#1,rectangle split horizontal,rectangle split part align=bottom}}
\tikzset{SPLIT/.style={rectangle split,rectangle split parts=#1,rectangle split part align=left}}
\tikzset{SPLITv/.style={SPLIT=#1,rectangle split horizontal=false}}
\tikzset{VAL/.style={val,TRIANGLE,font=\sf\small}}
\tikzset{VAR/.style={var,name=#1,append after command={node[above=0 of #1,font=\ttfamily] {#1}}}}


\tikzset{ground/.style={%
  shape=ground IEC,
  color=black,
  draw,
  ultra thick,
  name=#1,
}}

\tikzset{groundDown/.style={ground=#1, rotate=-90}}
\tikzset{groundUp/.style={ground=#1, rotate=90}}
\tikzset{groundLeft/.style={ground=#1, rotate=180}}
\tikzset{groundRight/.style={ground=#1, rotate=0}}


% Draw a "DOT" which usually denotes the source of a pointer. The argument
% is where this shape will be located. 
\newcommand\DT[2][]{\node[name=_,fill=black,circle,radius=2pt,inner sep=2pt,outer sep=0pt,#1] at (#2) {};}


% Draws a pointer emanating from it. The first argument is where 
%  
% is where this shape will be located. 
\newcommand\VHPTR[2]{\draw(#1) |- (#2);}


\bash[stdoutFile=\jobname.thapl.tex]
PYTHONPATH=".." python -m thapl.main ../Examples/garbage_collector.thapl
\END


\begin{document}

\tikzset{Vertical Record/.style={%
  CONSv=3,
  name=#1,
  append after command={%
    coordinate (#1 1st) at ($ (#1.north)!1/6!(#1.south)$)
    coordinate (#1 2nd) at ($ (#1.north)!5/6!(#1.south)$)
  }
}}

\tikzset{Horizontal Record/.style={%
  CONSh=3,
  name=#1,
  append after command={%
    coordinate (#1 1st) at ($ (#1.west)!1/6!(#1.east)$)
    coordinate (#1 2nd) at ($ (#1.west)!5/6!(#1.east)$)
    [draw,fill=red] (#1 1st) circle (2pt)
    [draw,fill=red] (#1 2nd) circle (2pt)
  },
}}

\def\verticalRecord=#1[#2]{%
  \node[Vertical Record = #1,#2] {};
  \DT{#1 1st}
  \DT{#1 2nd}
}

\def\horizontalRecord=#1[#2]{%
  \node[Horizontal Record=#1,#2] {};
  \DT{#1 1st}
  \DT{#1 2nd}
}

\tikzset{Scalar Variable/.style={var,name=#1,pattern=grid,pattern color=red}}
\tikzset{cell/.style={minimum width=0.9cm,minimum height=0.5cm,rectangle,thin}}

\tikzset{>= triangle 45}
\tikzset{shorten >=3pt}

\begin{frame}[fragile,t]{Mark \& sweep garbage collection}
  \begin{adjustbox}{}
    \begin{tikzpicture}
      \newcommand\firstLevel{}
      \newcommand\firstLevelArrow{}
      \newcommand\secondLevel{}
      \newcommand\secondLevelArrow{}
      \newcommand\thirdLevel{}
      \newcommand\thirdLevelArrow{}
      \newcommand\fourthLevel{}
      \tikzstyle{highlight}=[draw=red,ultra thick]
      \tikzstyle{highlightArrow}=[draw=red,thick]
        \node(store) [
          draw,
          minimum width=20cm,
          minimum height=10cm,
          fill=yellow!30,
        ] at (0,0) {};
        \node(heap) [
          draw,
          minimum width=18.5cm,
          minimum height=1.5cm,
          fill=blue!10,
          yshift=-3.5cm,
          label=below:{\bf\large Heap Data Structure},
        ] at (store) {};
        \begin{scope}[
            start chain=going left,
            node distance=.4cm,
            every node/.style={%
              on chain,
              join,
              fill=green!50!black,
              minimum width=2mm,
              minimum height=3.5mm,
              rectangle,
            },
            every join/.style={->,>= stealth,shorten >=1pt}
          ]
          \node[xshift=-4.ex,yshift=-0.25cm] at (heap.east) {};
          \node[minimum width=4mm]{};
          \node[minimum width=8mm]{};
          \node[minimum width=10mm](free){};
          \node[minimum width=2mm]{};
          \node[minimum width=14mm]{};
          \node[minimum width=7mm]{};
          \node[minimum width=20mm]{};
            \node[minimum width=2mm]{};
            \node[minimum width=2mm]{};
            \node[minimum width=2mm]{};
            \node[minimum width=2mm]{};
        \end{scope}
          \node[above=1mm of free,text=green!50!black]{\bf\large List of Free Blocks};
        \begin{scope}[
            start chain,
            node distance=.4cm,
            every node/.style={%
              on chain,
              join,
              fill=blue!50,
              minimum width=5mm,
              circle
            },
            every join/.style={<-,>= stealth,shorten <=1pt}
          ]
          % \uncover<1-52,56->{%
          \node[xshift=4.ex,yshift=0.25cm] at (heap.west) (a1) {};
          \node (a2) {};
          % }
          \node (a3) {};
          \node (a4) {};
          \node (a5) {};
          \node (a6) {};
            % \invisible<53-55>{%
            \node (a7) {};
            \node (a8) {};
          \node (a9) {};
          \node (a10) {};
          \node (a11) {};
          \node (a12) {};
        \end{scope}
      {\draw[->,>= stealth,shorten <= 1pt](a9)--(a6);}
      \begin{scope}[every node/.style={shape=circle,minimum width=5mm,fill=teal!50!}]
          {\node[GREEN] at (a1) {};}
          {\node at (a1) {};}
          {\node[GREEN] at (a2) {};}
          {\node at (a2) {};}
        {\node at (a3) {};}
        {\node at (a4) {};}
        {\node at (a5) {};}
        {\node at (a6) {};}
          {\node[GREEN] at (a7) {};}
          {\node at (a7) {};}
          {\node[GREEN] at (a8) {};}
          {\node at (a8) {};}
        {\node at (a9) {};}
        {\node at (a10) {};}
        {\node at (a11) {};}
        {\node at (a12) {};}
      \end{scope}
          \DT{a1}
          \DT{a2}
        \DT{a3}
        \DT{a4}
        \DT{a5}
        \DT{a6}
          \DT{a7}
          \DT{a8}
        \DT{a9}
        \DT{a10}
        \DT{a11}
        \DT{a12}
        \node[below=1mm of a3]{\bf\large List of Allocated Cells};
      \node(root) [
          draw,
          minimum width=16cm,
          minimum height=2.5cm,
          fill=red!10,
          xshift=0,
          yshift=2.7cm,
          label=above:{\Large\bf Root Set}
        ] at (store) {};
      \node(stack) [
        yshift=-0.2cm,
        xshift=3.5cm,
      ]{};
      \begin{scope}[% The Stack
          start chain=going left,
          node distance=0mm,
          every node/.style={on chain},
        ]
          \node[Scalar Variable=scalar 1,xshift=-3ex] at (root.east) {};
          \node[var,\firstLevel] (stack 1){};\DT{stack 1}
          \node[Scalar Variable=scalar 2]{};
          \node[var] (dots 1){$·⋯·$};
          \node[Scalar Variable=scalar 3] {};
          \node[var,\firstLevel] (stack 2){};\DT{stack 2}
          \node[Scalar Variable=scalar 4]{};
          \node[fill=red!30,inner sep=0,minimum height=0.5cm] (dots 2){$·⋯⇐$};
      \end{scope}
        \node[above=0 of dots 1,text=red,font=\bf\Large] {Runtime Stack};
      % Global Variables
        \node[Scalar Variable=a] at ($ (root.west)!.3!(root.east)$) {};
        \node[Scalar Variable=b,yshift=-2] at ($ (a)!.5!(root.west)$) {};
        \node(c) [var,yshift=-6,\firstLevel] at (root.center) {};\DT{c}
        \node(d) [var,yshift=-3,\firstLevel] at ($ (a)!.5!(c)$) {}; \DT{d}
        \node(e) [var,yshift=-6,\firstLevel] at ($ (b)!.5!(a)$) {}; \DT{e}
        \node(f) [var,yshift=-6,\firstLevel] at ($ (b)!.5!(root.west)$) {}; \DT{f};
        \node(global) [
          draw,
          label=above:{{\large\bf Global Variables}},
          color=red,
          cloud,
          cloud puffs=30,
          cloud puff arc=120,
          inner sep=0,
          minimum width=9.5cm,
          minimum height=1.6cm,
          yshift=-6,
          xshift=4
        ] at (a) {};
      % Heap Variables
        \horizontalRecord=v1[below=10ex of stack 2,\secondLevel]
        \horizontalRecord=v2[below=4ex of v1 1st,\thirdLevel]
        \verticalRecord=v3[below=18ex of c,\secondLevel]
        \verticalRecord=v4[right=of v3,\thirdLevel]
        \verticalRecord=x1[below=16ex of d]
        \node[Horizontal Record=x2,
          below=14ex of global.east,
          pattern=grid,
          pattern color=blue,
        ]{};
        \horizontalRecord=c1[below=12ex of e,\secondLevel]
        \verticalRecord=c2[below=15ex of e,xshift=8ex,\thirdLevel]
        \verticalRecord=c4[below=15ex of e,xshift=-8ex,\thirdLevel]
        \horizontalRecord=x3[below=14ex of global.west,xshift=-3ex
        ]
        \horizontalRecord=x4[below=of x3]
        \horizontalRecord=c3[below=8ex of c1,\fourthLevel]
      % Null pointers
        \path let \p1 = (v1.one), \p2 = (root.east) in node[groundRight=g1,yshift=2] at (\x2,\y1){};
        \node[groundRight=g2,below=4ex of g1,yshift=4pt] {};
        \path let \p1 = (v4.one), \p2 = (g2) in node[groundRight=g3,yshift=2] at (\x2,\y1){};
        \node[groundRight=g4, below=4ex of g3,yshift=4pt]{};
        \node[groundDown=g5,below=12ex of d,yshift=4pt]{};
        \node[groundLeft=G1,left=12ex of f.west] {};
        \node[groundRight=G2,below=31 ex of G1,xshift=6ex] {};
        \draw[->](f)->(G1);
        \draw[->](v1 2nd)->(g1.input);
        \VHPTR{stack 1}{g1.input}
        \draw[->](v2 2nd)->(g2.input);
        \VHPTR{v2 1st}{g3}
        \draw[->](v4 1st)->(g3);
        \draw[->](v4 2nd)->(g4);
        \draw[->](d.center)->(g5.input);
        \draw[->] (x4 1st) |- (G2);
          \draw[->,bend left= 45] (a1) to (x3.west);
          \draw[->,bend right= 60] (a2) to (x4.east);
          \draw[->,bend left=120] (a3)..controls(f) and ($ (c1.north)!0.55!(f)$)..(c1.north west);
          \draw[->,bend right=30] (a4)to(c4.south);
          \draw[->,bend right=20] (a5)to(c3.south);
          \draw[->,bend right=20] (a6)to(c2.south);
          \draw[->,bend right=20] (a7)to(x1.south);
          \draw[->](a8)..controls ($ (x1.south east)!0.5!(v3.south west)$)
          and ($ (x1.north east)!0.4!(v3.north west)$)..(x2.west);
          \draw[->] (a9) to (v3.west);
          \draw[->] (a10) to (v4.west);
          \draw[->] (a11)..controls(g2) and (g1)..(v2.east);
          \draw[->] (a12)..controls(g4) and ($ (store.north east)!0.8!(g1)$)..(v1.east);
      % Intra-variables references
        \draw[->,\firstLevelArrow](stack 2)->(v1);
        \draw[->,\firstLevelArrow](c)->(v3.north);
        \draw[->,\firstLevelArrow](e)->(c1.north);
        %
        \draw[->,\secondLevelArrow](v1 1st)->(v2.north);
        \draw[->,\secondLevelArrow](v3 1st) to (v4 1st);
        \draw[->,\secondLevelArrow](v3 2nd) to (v4 2nd);
        \draw[->,\secondLevelArrow,bend left=30](c1 2nd) to (c2.north);
        %
        \draw[->,\secondLevelArrow,bend left=30](c1 1st) to (c4.east);
        \draw[->,\thirdLevelArrow,bend left=30](c2 2nd) to (c3.east);
        \draw[->,\thirdLevelArrow,bend left=30](c4 2nd) to (c3.north);
        \draw[->,\thirdLevelArrow,bend left=30](c3 1st) to (c4.south);
        %
        \draw[->,bend left=30](c2 1st) to (c1.south);
        \draw[->,bend left=30](c3 2nd) to (c2.west);
        \draw[->,bend left=30](c4 1st) to (c1.west);
        \draw[->,bend right=30](x3 1st) to (x4.west);
        \draw[->,bend right=30](x4 2nd) to (x3.east);
        \draw[->] (x3 2nd) to (f.south);
    \end{tikzpicture}
  \end{adjustbox}

\end{frame}


\include{\jobname.thapl}

\end{document}
